\documentclass{article}


\usepackage[dvipsnames]{xcolor}
\usepackage{amsmath}
\usepackage{amsfonts} 
\usepackage{tikz}
\usepackage{tkz-euclide}
\usepackage[russian]{babel} 
\usepackage{fancyhdr}
\usepackage{pgfplots}
\usepackage{algorithm2e}
\usepackage{indentfirst}
\usepackage[rightcaption]{sidecap}
\usepackage{subcaption}
\usepackage{wrapfig}
\usepackage{float}


\pagecolor{black}
\color{white}


\title{\textbf{СВТ}\\ \textbf{Численное решение 1D-уравнения Лапласа} }
\author{Арсений Е. Грознецкий}
\date{21-е марта 2025 года}


\begin{document}

\maketitle
\tableofcontents

\newpage

\section{Постановка задачи}

Требуется численно решить следующую краевую задачу Дирихле:

$$
\begin{cases}
-u''(x) = f(x), \quad x \in (0; 1) \\
u(0) = a, \: u(1) = b
\end{cases}
$$

Для этого на отрезке $(0; 1)$ вводится равномерная сетка $\{x_0, x_1, \ldots, x_N\}$, где $x_i = ih$, $h = 1 / N$ - шаг сетки. Обозначив $y_i = u(x_i)$ получим дискретное уравнение, приближающее уравнение Лапласа:
$$-\frac{y_{i-1} -2y_i + y_{i+1}}{h^2} = f(x_i)$$
Уравнения образуют следующую систему:
$$\frac{1}{h^2}
\begin{bmatrix}
+2 & -1   &      &      &     \\
-1 & +2   &  -1  &      &     \\
   &\ddots&\ddots&\ddots&     \\
   &      &  -1  &  +2  &  -1 \\
   &      &      &  -1  &  +2 \\
\end{bmatrix} = \begin{bmatrix}
y_1 \\
y_2 \\ 
\vdots \\
y_{N-2} \\
y_{N-1} \\
\end{bmatrix} = \begin{bmatrix}
f(x_1) + a / h^2 \\
f(x_2) \\ 
\vdots \\
f(x_{N-2}) \\
f(x_{N-1}) + b / h^2 \\
\end{bmatrix} $$

Системы с трёхдиагональными матрицами можно решать методом прогонки.


\section{Прогонка}

Система уравнений \( Ax = F \) равносильна соотношению:
\begin{equation}
    A_i x_{i-1} + B_i x_i + C_i x_{i+1} = F_i.
\end{equation}
Здесь $A_i, B_i, C_i$ - элементы нижней, главной и верхней диагоналей соответственно.
Метод прогонки основывается на предположении, что искомые неизвестные связаны рекуррентным соотношением:
\begin{equation}
    x_i = \alpha_{i+1} x_{i+1} + \beta_{i+1}, \quad i = n-1, n-2, \dots, 1.
\end{equation}

Тогда выразим \( x_{i-1} \) и \( x_i \) через \( x_{i+1} \) и подставим в уравнение (1):
\begin{equation}
    (A_i \alpha_i \alpha_{i+1} + B_i \alpha_i + C_i) x_{i+1} + A_i \alpha_i \beta_{i+1} + A_i \beta_i + B_i \beta_{i+1} - F_i = 0,
\end{equation}
где \( F_i \) — правая часть \( i \)-го уравнения. Это соотношение будет выполняться независимо от решения, если потребовать:
\begin{equation}
    \begin{cases}
        A_i \alpha_i \alpha_{i+1} + B_i \alpha_i + C_i = 0, \\
        A_i \alpha_i \beta_{i+1} + A_i \beta_i + B_i \beta_{i+1} - F_i = 0.
    \end{cases}
\end{equation}

Отсюда следует:
\begin{equation}
    \begin{cases}
        \alpha_{i+1} = \frac{-C_i}{A_i \alpha_i + B_i}, \\
        \beta_{i+1} = \frac{F_i - A_i \beta_i}{A_i \alpha_i + B_i}.
    \end{cases}
\end{equation}

Из первого уравнения получим:
\begin{equation}
    \begin{cases}
        \alpha_2 = \frac{-C_1}{B_1}, \\
        \beta_2 = \frac{F_1}{B_1}.
    \end{cases}
\end{equation}

После нахождения прогонных коэффициентов \( \alpha \) и \( \beta \), используя уравнение (2), получим решение системы. При этом:
\begin{equation}
    x_i = \alpha_{i+1} x_{i+1} + \beta_{i+1}, \quad i = n-1, \dots, 1.
\end{equation}
\begin{equation}
    x_n = \frac{F_n - A_n \beta_n}{B_n + A_n \alpha_n}.
\end{equation}

Пользуясь этим методом, решим систему при разных $N$.

\section{Численные результаты}

Задача Дирихле решалась для $u(x) = \sin(\pi x)$, то есть $f(x) = \pi^2\sin(\pi x)$. Резултаты экспериментов представлены на графике \ref{fig1} ниже.


\begin{figure}[h!]
    
    \begin{minipage}{.50\textwidth}
    
        \begin{tikzpicture}
        
            \begin{loglogaxis} [
                draw=white,text=white,
                legend pos = south west,
                legend style={draw=black,fill=black,text=white},
                width = 175, height = 175,
                xmin = 10, xmax = 100000, domain = 10:100000,
                ymin = 5.855370843054471e-08, ymax = 0.01848203419886702,
                xlabel = {$N$}, ylabel = {$error(N)$},
                every axis x label/.style={at={(current axis.north)},yshift=2.5mm},
            ]
            
            \addplot[green] table [col sep=comma]{./data/l2.csv};
            
            \end{loglogaxis}
            
        \end{tikzpicture}
    
        \subcaption{L-2 метрика}
        \label{fig:unstable}
        
    
    \end{minipage}
    \begin{minipage}{.50\textwidth}

        \begin{tikzpicture}
        
            \begin{loglogaxis} [
                draw=white,text=white,
                legend pos = south west,
                legend style={draw=black,fill=black,text=white},
                width = 175, height = 175,
                xmin = 10, xmax = 100000, domain = 10:100000,
                ymin = 3.3081826167347117e-10, ymax = 0.008265416966228623,
                xlabel = {$N$}, ylabel = {$error(N)$},
                every axis x label/.style={at={(current axis.north)},yshift=2.5mm},
                every axis y label/.style={at={(current axis.east)},rotate=90, yshift=-2.5mm},
            ]
            
            \addplot[green] table [col sep=comma]{./data/c.csv};
            
            \end{loglogaxis}
        
        \end{tikzpicture}
        
        \subcaption{Чебышевская метрика}
        \label{fig:stable}
    
    \end{minipage}

    \caption{Сходимость решения задачи Дирихле по разным метрикам.}
    \label{fig1}
    
\end{figure}


\section{Выводы}

В данной задаче реализовали алгоритм решения задачи Дирихле, и убедились в его численной сходимости.


\end{document}
